\documentclass{article}
\usepackage{mathpartir}
\usepackage{tikz-cd}
\usepackage{enumitem}
\usepackage{wrapfig}
\usepackage{fancyvrb}
\usepackage{comment}

\usepackage{ stmaryrd }


\newcommand{\Set}{\textrm{Set}}

\begin{document}

\title{First-order Hyperdoctrines}

We give an alternate definition of first-order hyperdoctrines that
makes the Beck-Chevalley condition follow naturally. The key is to
formulate the definitions using the internal language of the category
of presheaves. This is based on the fact that, like most algebraic
structures a functor from $C$ to the category of Heyting algebras is
equivalent to a Heyting algebra internal to the category of functors
$C \to \Set$


\section{Order-theory preliminaries}

Let $f : P \to Q$ be a monotone function of posets. A \textbf{right
  adjoint} to $f$ is a function $g : Q_0 \to P_0$ such that
\[ p \leq g(q) \iff f(p) \leq q \]
A \textbf{left adjoint} is defined dually as satisfying
\[ g(q) \leq p \iff q \leq f(p) \]
Note that in either case $g$ is unique if it exists. The analogous
notion for preorders is unique up to order equivalence.

For any poset $P$ and type $X$ we can define the poset $P^X$ of
functions $X \to P_0$ with the pointwise ordering. We can define a
monotone function $\Delta_X : P \to P^X$
\[ \Delta_X(p)(x) = p \]
which we could sensibly call ``weakening''.
We say $P$ has \textbf{$X$-indexed meets} if $\Delta_X$ has a right
adjoint and \textbf{$X$-indexed joins} if $\Delta_X$ has a left
adjoint. Examples: binary meets and joins are indexed by a two-element
set and top and bottom are indexed by the empty set.

Not all properties are naturally formulated as adjoints. The following
is a generalization.  Let $P$ be a poset. A \emph{downward-closed
subset} or \emph{downset} is a subset $X \subseteq P$ such that if $p
\leq q$ and $q \in X$ then $p \in X$. An upper closed set/upset is
dual. A \emph{representing element} of a downset is a greatest element
of the downset and a representing element of an upset is a least
element of the upset. In this case we say the downset/upset is
\emph{representable}, the idea is that if $x \in X$ is the greatest
element of the downset $X$ .
\[ p \in X \iff p \leq x \]
so $x$ ``represents'' by its inequality predicate the subset.

Let $P$ be a poset with a greatest element $\top$ and $X$ be a
set. Then an \emph{equality function for $X$ in $P$} is a function $=$
in $(P^X)^X$ satisfying
\[ (\forall x,y. (x = y) \leq f(x,y)) \iff \forall x. \top \leq f(x,x) \]
the Leibniz/Lawvere formulation of equality. This is a
representability condition for the upset defined as
\[ f(x,y) \in \textrm{U} \iff \forall x. \top \leq f(x,x) \]

\section{Internal Order Theory}

Let $C$ be a category, then the category of presheaves $\Set^{C^{op}}$
inherits almost all nice properties of the category of sets. One way
to say this is that it is always a \emph{topos}, a model of
intuitionistic type theory. So order theory internal to presheaves
means ``just'' interpret everything in the previous section in this
model of intuitionistic type theory.

TODO: explain internal order theory

\section{First-Order Hyperdoctrines}

A first-order hyperdoctrine over a cartesian category $C$ consists of
\begin{enumerate}
\item (Propositional Logic) A Heyting algebra $L$ internal to $\Set^{C^{op}}$
\item (Universal Quantifiers) Such that for every $A \in C$, $L$ internally has $YA$-indexed meets
\item (Existential Quantifiers) Such that for every $A \in C$, $L$ internally has $YA$-indexed joins
\item (Equality) Such that $L$ has an internal equality function for $YA$
\end{enumerate}

\section{Syntax}

TODO: this is all very syntactic...

\section{Grothendieck Construction}

TODO: logical relations...

\end{document}
